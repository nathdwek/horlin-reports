\subsection{Introduction}


\subsection{Question}
\subsubsection{Simulation}

\paragraph{Derive analytically the baseband model of the channel including the synchronisation errors.} Synchronization errors contains :
\begin{itemize}
\item a Carrier Frequency Offset (CFO) $\Delta \omega$
\item a phase offset $\phi_0$
\item a Sample Clock Offset (SCO) $\delta$
\item a time shift $t_0$
\end{itemize}
From these, one can update the baseband model as shown in figure~\ref{fig:sync}.

\begin{figure}[htbp]
\includegraphics[width=\textwidth]{baseband_sync.png}
\caption{Baseband model with synchronization errors.\label{fig:sync}}
\end{figure}


\paragraph{How do you separate the impact of the carrier phase drift and ISI due to the CFO in your simulation?}

\paragraph{How do you simulate the sampling time shift in practice?} In our simulation, the sampling time shift is simulate by  truncation of the received signal.

\paragraph{How do you select the simulated $E_{b}/N_{o}$ ratio?} Typical value of \SI{10}{\deci\bel} ? small enough to get something correct at the end and big enough to get some errors to be able to test our simulated channel with noise.

\paragraph{How do you select the lengths of the pilot and data sequences?} The pilot's length should be long enough to get a good estimation of the phase and the length of the data is selected to ensure a correct phase interpolation between two pilot sequences. 

\subsubsection{Communication System}


\paragraph{In which order are the synchronisation effects estimated and compensated. Why?} tips : it remains a small CFO who creates a linearly varying  phase over the time


\paragraph{Explain intuitively how the error is computed in the Gardner algorithm. Why is the
Gardner algorithm robust to CFO?} The algorithm works with a feedback structure. Each time shift estimate is the sum of the previous one and a weigthed version of the error. With a good selection of coefficient, the error on the phase converges to zero ... 


\paragraph{Explain intuitively why the differential cross-correlator is better suited than the usual cross-correlator? Isn’t interesting to start the summation at $k = 0$ (no time shift)?}


\paragraph{Are the frame and frequency acquisition algorithms optimal? If yes, give the optimisation criterion.}

